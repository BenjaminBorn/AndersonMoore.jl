\documentclass[12pt]{article}

\usepackage{dirtree}
\usepackage{hyperref}
\usepackage{amsmath}
\title{Notes for Julia Implementation of the Anderson-Moore Algorithm}
\author{Gary S. Anderson}
\begin{document}
\maketitle
\begin{abstract}
  This project will develop a Julia version of the Anderson-Moore Algorithm to add to the \href{https://www.federalreserve.gov/econres/ama-index.htm}{FEDS website}  and to contribute to the \href{http://quantecon.org/quantecon.jl}{quantecon initiative.}  
%The intern will also help develop components of a Julia implementation of a new series representation of discrete time maps useful for solving nonlinear models ith occasionally binding constraints and regime-switching.
\end{abstract}
\begin{description}
\item[Algorithm Resources] \ 
\begin{itemize}
\item \href{https://github.com/es335mathwiz/SPSolve.git}{SPSolve -- MATLAB implementation of the algorithm}
\item \href{https://github.com/es335mathwiz/sparseAMA.git}{sparseAMA -- ``C'' implementation of the algorithm}
\item \href{https://github.com/es335mathwiz/mathAMA.git}{Mathematica implementation}
\item \href{https://www.federalreserve.gov/econres/ama-index.htm}{FED's AMA site}
\item \href{https://www.sciencedirect.com/science/article/pii/S1474667017405064}{``A Reliable and Computationally Efficient Algorithm for Imposing the Saddle Point Property in Dynamic Models''}
\item \href{https://www.sciencedirect.com/science/article/pii/0165176585902113}{``A linear algebraic procedure for solving linear perfect foresight models''}
\end{itemize}
\item[Julia Resources] \
  \begin{itemize}
  \item \href{https://docs.julialang.org/en/stable/}{Documentation}
  \item Base.qr or \href{https://github.com/JuliaMatrices/LowRankApprox.jl}{LowRankApprox.jl} for QR decomposition?
  \item \href{http://www.stochasticlifestyle.com/finalizing-julia-package-documentation-testing-coverage-publishing/}{Finalizing a Julia Package}
  \item \href{https://www.google.com/url?sa=t&rct=j&q=&esrc=s&source=web&cd=4&cad=rja&uact=8&ved=0ahUKEwjh_vH_h-DaAhUExYMKHYb7BWoQFgg5MAM&url=https%3A%2F%2Fgithub.com%2Ftonyhffong%2FLint.jl%2Fblob%2Fmaster%2Fdocs%2Ffeatures.md&usg=AOvVaw1FFQ9J64mhTjDx8QpQ-_zU}{Lint.jl}
 \item \href{https://github.com/JuliaIO/MAT.jl}{read matlab .mat files}
 \item on our linux, use  {\bf julia-0.6 }
  \end{itemize}
\newpage
\item[git Resources] \ 

  \begin{itemize}
  \item \href{https://tutorialzine.com/2016/06/learn-git-in-30-minutes}{tutorialzine article}
  \item \href{https://try.github.io/}{Resources to learn git}
  \item \href{https://www.atlassian.com/git}{Getting Git Right}
  \end{itemize}

\item[\LaTeX Resources] \ 

  \begin{itemize}
  \item \href{https://www.sharelatex.com/learn/Learn_LaTeX_in_30_minutes}{learn in 30 minutes}
  \item \href{https://www.tug.org/begin.html}{getting started with \LaTeX}
  \end{itemize}
\item[Code] \
  
  \begin{itemize}
  \item lowerCamelCase
  \item variable names searchable( ii instead of i )
  \end{itemize}

\dirtree{%
.1 AMAalg.
.2 exactShift.
.3 shiftRight.
.2 numericShift.
.3 shiftRight.
.2 buildA.
.2 eigenSystem.
.2 augmentQ.
.2 reducedForm.
}

\item[AMAalg] The algorithm

\item[exactShift] shift rows of matrix to the right in an effort to create a nonsingular rightmost block. This routine detects shiftable rows before any matrix multiplication
\item[shiftRight]  does the actual shifting 
\item[numericShift] shift rows of matrix to the right in an effort to create a nonsingular rightmost block. This routine uses the QR decomposition to create  more shiftable rows
\item[shiftRight] does the actual shifting 
\item[buildA] builds the transition matrix
\item[eigenSystem] computes the eigenvalues and the vectors spanning the left invariant space using eigs (arpack)
\item[augmentQ] combines the shifted rows and the left invariant space spanning vectors
\item[reducedForm] computs the ``B'' matrix reduced form representation of the solution



\end{description}



\end{document}
